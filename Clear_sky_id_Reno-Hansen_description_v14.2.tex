% Options for packages loaded elsewhere
\PassOptionsToPackage{unicode}{hyperref}
\PassOptionsToPackage{hyphens}{url}
\PassOptionsToPackage{dvipsnames,svgnames,x11names}{xcolor}
%
\documentclass[
  10pt,
  a4paper,oneside]{article}
\usepackage{amsmath,amssymb}
\usepackage{iftex}
\ifPDFTeX
  \usepackage[T1]{fontenc}
  \usepackage[utf8]{inputenc}
  \usepackage{textcomp} % provide euro and other symbols
\else % if luatex or xetex
  \usepackage{unicode-math} % this also loads fontspec
  \defaultfontfeatures{Scale=MatchLowercase}
  \defaultfontfeatures[\rmfamily]{Ligatures=TeX,Scale=1}
\fi
\usepackage{lmodern}
\ifPDFTeX\else
  % xetex/luatex font selection
\fi
% Use upquote if available, for straight quotes in verbatim environments
\IfFileExists{upquote.sty}{\usepackage{upquote}}{}
\IfFileExists{microtype.sty}{% use microtype if available
  \usepackage[]{microtype}
  \UseMicrotypeSet[protrusion]{basicmath} % disable protrusion for tt fonts
}{}
\makeatletter
\@ifundefined{KOMAClassName}{% if non-KOMA class
  \IfFileExists{parskip.sty}{%
    \usepackage{parskip}
  }{% else
    \setlength{\parindent}{0pt}
    \setlength{\parskip}{6pt plus 2pt minus 1pt}}
}{% if KOMA class
  \KOMAoptions{parskip=half}}
\makeatother
\usepackage{xcolor}
\usepackage[left=0.5in,right=0.5in,top=0.5in,bottom=0.5in]{geometry}
\usepackage{longtable,booktabs,array}
\usepackage{calc} % for calculating minipage widths
% Correct order of tables after \paragraph or \subparagraph
\usepackage{etoolbox}
\makeatletter
\patchcmd\longtable{\par}{\if@noskipsec\mbox{}\fi\par}{}{}
\makeatother
% Allow footnotes in longtable head/foot
\IfFileExists{footnotehyper.sty}{\usepackage{footnotehyper}}{\usepackage{footnote}}
\makesavenoteenv{longtable}
\usepackage{graphicx}
\makeatletter
\def\maxwidth{\ifdim\Gin@nat@width>\linewidth\linewidth\else\Gin@nat@width\fi}
\def\maxheight{\ifdim\Gin@nat@height>\textheight\textheight\else\Gin@nat@height\fi}
\makeatother
% Scale images if necessary, so that they will not overflow the page
% margins by default, and it is still possible to overwrite the defaults
% using explicit options in \includegraphics[width, height, ...]{}
\setkeys{Gin}{width=\maxwidth,height=\maxheight,keepaspectratio}
% Set default figure placement to htbp
\makeatletter
\def\fps@figure{htbp}
\makeatother
\setlength{\emergencystretch}{3em} % prevent overfull lines
\providecommand{\tightlist}{%
  \setlength{\itemsep}{0pt}\setlength{\parskip}{0pt}}
\setcounter{secnumdepth}{-\maxdimen} % remove section numbering
\usepackage{caption}
\usepackage{placeins}
\captionsetup{font=small}
\usepackage{multicol}
\setlength{\columnsep}{1cm}
\ifLuaTeX
  \usepackage{selnolig}  % disable illegal ligatures
\fi
\IfFileExists{bookmark.sty}{\usepackage{bookmark}}{\usepackage{hyperref}}
\IfFileExists{xurl.sty}{\usepackage{xurl}}{} % add URL line breaks if available
\urlstyle{same}
\hypersetup{
  pdftitle={Identification of Periods of Clear Sky Irradiance in Time Series of GHI Measurements Matthew J. Reno and Clifford W. Hansen.},
  pdfauthor={Natsis Athanasios},
  colorlinks=true,
  linkcolor={Maroon},
  filecolor={Maroon},
  citecolor={Blue},
  urlcolor={Blue},
  pdfcreator={LaTeX via pandoc}}

\title{Identification of Periods of Clear Sky Irradiance in Time Series of GHI Measurements Matthew J. Reno and Clifford W. Hansen.}
\author{Natsis Athanasios}
\date{}

\begin{document}
\maketitle

{
\hypersetup{linkcolor=}
\setcounter{tocdepth}{4}
\tableofcontents
}
\newpage

\hypertarget{detection-of-clear-periods-in-ghi-measurements-for-sdr-tredns.}{%
\section{Detection of clear periods in GHI measurements for SDR tredns.}\label{detection-of-clear-periods-in-ghi-measurements-for-sdr-tredns.}}

\hypertarget{only-filters-for-ghi-are-used-for-sdr-tredns}{%
\subsection{Only filters for GHI are used for SDR tredns}\label{only-filters-for-ghi-are-used-for-sdr-tredns}}

\hypertarget{load-all-data-from-databroad_bandqcrad_longshiqcrad_longshi_v8_apply_cm21_chp1_0-94.rds}{%
\subsection{\texorpdfstring{Load all data from \texttt{DATA/Broad\_Band/QCRad\_LongShi/QCRad\_LongShi\_v8\_apply\_CM21\_CHP1\_{[}0-9{]}\{4\}.Rds}}{Load all data from DATA/Broad\_Band/QCRad\_LongShi/QCRad\_LongShi\_v8\_apply\_CM21\_CHP1\_{[}0-9{]}\{4\}.Rds}}\label{load-all-data-from-databroad_bandqcrad_longshiqcrad_longshi_v8_apply_cm21_chp1_0-94.rds}}

\hypertarget{exclude-data-where-wattglb-watthor}{%
\subsection{\texorpdfstring{Exclude data where \texttt{wattGLB\ \textless{}\ wattHOR}}{Exclude data where wattGLB \textless{} wattHOR}}\label{exclude-data-where-wattglb-watthor}}

There are instances where global irradiance is less than direct.

This happens, near sunset and sunrise due to obstacles,
or due to different sunrise/sunset time due to small spatial differences.
We will exclude all this data both for global and direct.

\hypertarget{exclude-other-bad-data-indentified-in-raw-data}{%
\subsection{Exclude other bad data indentified in raw data}\label{exclude-other-bad-data-indentified-in-raw-data}}

\hypertarget{threshold-values-for-criteria}{%
\subsubsection{Threshold Values for Criteria}\label{threshold-values-for-criteria}}

For mean and max Most evaluations of clear sky models find that the average
bias error of the model is less than 10\%, often around 7\% {[}1, 83{]}. Therefore,
a fixed threshold of \(\pm 75 W / m^2\) within the mean and max of the clear sky model
was chosen.

\hypertarget{clear-sky-detection-algorithm-values}{%
\subsubsection{Clear Sky detection Algorithm values}\label{clear-sky-detection-algorithm-values}}

\hypertarget{reference-clear-sky-irradiance-model-and-air-mass-model-am}{%
\subsubsection{Reference clear sky irradiance model and air mass model (AM)}\label{reference-clear-sky-irradiance-model-and-air-mass-model-am}}

We select the clear sky model, the air mass calculation
and the period of 11 minutes to use.

\hypertarget{low-direct-irradiance-limit-ldi}{%
\subsubsection{6. Low Direct Irradiance limit (LDI)}\label{low-direct-irradiance-limit-ldi}}

Ignore Direct irradiance when it is bellow 5 \(Watt/m^2\).
This limit is biased due to the slight location difference of the two Instruments.
The difference in shadow especially the afternoon ``pole shadow'' of some periods of the year.
May limit the implementation by time of day or/and period of the year

Also this can not characterize all global measurements due to gaps on direct measurements.

\hypertarget{low-global-irradiance-limit-lgi}{%
\subsubsection{7. Low Global Irradiance limit (LGI)}\label{low-global-irradiance-limit-lgi}}

\hypertarget{too-few-cs-point-for-the-day-fcs}{%
\subsubsection{8. Too Few CS point for the day (FCS)}\label{too-few-cs-point-for-the-day-fcs}}

If in a day there less than 11 clear sky points will exclude them from optimizing

\hypertarget{too-few-data-points-for-the-day-fdp}{%
\subsubsection{9. Too Few data points for the day (FDP)}\label{too-few-data-points-for-the-day-fdp}}

If in a day there are less than 33 data points will be excluded from optimizing.

\hypertarget{too-low-direct-signal-dst}{%
\subsubsection{11. Too low Direct signal (DsT)}\label{too-low-direct-signal-dst}}

This is a simple hard limit on the lower possible Direct radiation with clear sky.

The threshold is 25\% lower than a reference value.

\[ I_d = I_0 * 0.7^{{AM}^{0.678}} * cos({ZSA}) \]

Where \({AM}\) is the selected air-mass model.

``Clear sky direct normal irradiance estimation based on adjustable inputs and error correction\_Zhu2019.pdf''

\hypertarget{baseline-value-for-direct-irradiance}{%
\subsubsection{Baseline value for direct irradiance}\label{baseline-value-for-direct-irradiance}}

See: Clear sky direct normal irradiance estimation based on adjustable inputs and error correction\_Zhu2019.pdf

\begin{longtable}[]{@{}
  >{\centering\arraybackslash}p{(\columnwidth - 2\tabcolsep) * \real{0.1250}}
  >{\raggedright\arraybackslash}p{(\columnwidth - 2\tabcolsep) * \real{0.8750}}@{}}
\toprule\noalign{}
\begin{minipage}[b]{\linewidth}\centering
CS Flag
\end{minipage} & \begin{minipage}[b]{\linewidth}\raggedright
Test
\end{minipage} \\
\midrule\noalign{}
\endhead
\bottomrule\noalign{}
\endlastfoot
NA & Undefined, untested \\
0 & Passed as clear sky \\
1 & Mean value of irradiance during the time period (MeanVIP) \\
2 & Max Value of Irradiance during the time Period (MaxVIP) \\
3 & Variability in irradiance by the length (VIL) \\
4 & Variance of Changes in the Time series (VCT) \\
5 & Variability in the Shape of the irradiance Measurements (VSM) \\
6 & Low Direct Irradiance limit (LDI) \\
7 & Low Global Irradiance limit (LGI) \\
8 & Too Few CS point for the day (FCS) \\
9 & Too Few data points for the day \\
10 & Missing Data \\
11 & Direct irradiance simple threshold \\
\end{longtable}

\hypertarget{cost-function-for-optimization-of-alpha-value.}{%
\subsubsection{Cost function for optimization of alpha value.}\label{cost-function-for-optimization-of-alpha-value.}}

\[ f(a) = \dfrac{ \sum_{i=1}^{n} ( a \cdot {GHI}_i - {CSI}_i )^2 }
                { n } , \qquad a > 0 \]

\textbf{END}

\end{document}
